\documentclass[a4j,10pt]{jsarticle}

\usepackage{booktabs}
%grapic
\usepackage[dvipdfmx]{graphicx}

%sorce code
\usepackage{listings,jlisting}
%非標準パッケージ。エラー出るならコメントアウト。
%以下のようにする。
%\begin{lstlisting}[caption=キャプション,label=aaa]
%	コード...
%\end{lstlisting}   ←直書きする場合。
%\lstinputlisting[caption=キャプション,label=zzz]{filename}   ←ファイルから取り込む場合。

%listings, jlisting settings
%7行目をコメントアウトするなら、46行目までコメントアウトする。
\lstset{
	language=C,
	numbers=left,
	stepnumber=1,
	numbersep=1zw,
	numberblanklines=true,
	framexleftmargin=25pt,
	xleftmargin=25pt,
	captionpos=b,
	breaklines=true,
	breakatwhitespace=true,
	breakautoindent=true,
	emptylines=3,
	frame=tblr,
	tab=\textasciicircum\hspace{\fill},
	showtabs=true,
	tabsize=4,
	showspaces=false,
	lineskip=-0.5ex,
	numberstyle=\scriptsize,
	basicstyle=\tt\small,
	identifierstyle=\small,
	stringstyle=\tt,
	commentstyle=\small\itshape,
	keywordstyle=\small\bfseries,
	ndkeywordstyle=\scriptsize
}
%listingの図番号は通せなかった。
%代わりに「ファイル 1  キャプション」のように出力する。
\renewcommand{\lstlistingname}{ファイル}

%コマンド
%図
%\afigure{横幅}{ファイル名}{キャプション}{ラベル}
\newcommand{\afigure}[4][12cm]{
	\begin{center}
		\begin{figure}[htbp]
			\includegraphics[#1]{#2}
			\caption{#3}
			\label{#4}
		\end{figure}
	\end{center}
}

%使用器具
%\begin{aitems}{機種 or ソフトウェア}{ラベル}
%\aline{パソコン}{B60D}{ブランド}{自宅}
%\end{aitems}
\newenvironment{aitems}[2]{%
\begin{table}[htbp]
	\begin{center}
		\caption{使用#1}
		\label{#2}
		\begin{tabular}{l|l|c|l}
			\hline
			#1名 & 規格 & 製造会社 & その他 \\
			\hline
}{%
			\hline
		\end{tabular}
	\end{center}
\end{table}
}
\newcommand{\aline}[4]{#1 & #2 & #3 & #4 \\}

%セクション&ラベル
\newcommand{\mysection}[2]{\section{#1}\label{#2}}
\newcommand{\mysubsection}[2]{\subsection{#1}\label{#2}}
\newcommand{\mysubsubsection}[2]{\subsubsection{#1}\label{#2}}

%その他マクロ
%丸数字
\newcommand{\MARU}[1]{{\ooalign{\hfil#1\/\hfil\crcr
	\raise.167ex\hbox{\mathhexbox20D}}}}
%倍角ダッシュ
\def\――{―\kern-.5zw―\kern-.5zw―}
%斜線文字(数式モード)
\newcommand{\Slash}[1]{{\ooalign{\hfil/\hfil\crcr$#1$}}}
%大なり・小なりイコール
\makeatletter
\newcommand{\LEQQ}{\mathrel{\mathpalette\gl@align<}}
\newcommand{\GEQQ}{\mathrel{\mathpalette\gl@align>}}
\newcommand{\gl@align}[2]{\lower.6ex\vbox{\baselineskip\z@skip
	\lineskip\z@
	\ialign{$\m@th#1\hfil##\hfil$\crcr#2\crcr=\crcr}}}
\makeatother
%一行分数
\newcommand{\FRAC}[2]{\leavevmode\kern.1em
	\raise.5ex\hbox{\the\scriptfont0 #1}\kern-.1em
	/\kern-.15em\lower.25ex\hbox{\the\scriptfont0 #2}}
%数値羅列表環境(center + tabular環境の代替)
% \phantom{何々}で「何々」分の空白が入る。\rlap{何々}で、一部分だけ文字末尾にを足すこともできる(アスタリスクを一カ所だけ付加などに使う)。
\makeatletter
\newenvironment{arraytabular}[1]{%
	\begin{center}
		\begin{tabular}{#1}
}{%
		\end{tabular}
	\end{center}
}
\makeatother
%単位記号をローマン体にした、単位付き数値
\newcommand{\myvalue}[2]{$#1[\mathrm{#2}]$}


%ページ番号に関する設定
\pagestyle{myheadings}
\newif\ifpre
%プレレポートの時は\pretrue、本レポートでは\prefalse
\pretrue

%本文の開始
\begin{document}
\markboth{}{}
\ifpre\addtocounter{page}{1}\else
\begin{abstract}
\setlength{\baselineskip}{1.75zw}
$\FRAC{2}{5}$ここに概要を書く。両脇が空くのは\LaTeX 側の仕様なので、気になるようならば、\verb|\section*{概要}|
のように書く。
\end{abstract}
\newpage
\fi
\mysection{目的}{sec:purpose}

ここに目的を書く。


\mysection{原理}{sec:theory}

ここに原理を書く。文献は参考文献の節を参照\cite{アイテム}。複数文献を参照するときは、このようにする\cite{アイテム,新}。ページ番号も挿入する場合は、こうする\cite[pp.9--44]{新}。


\mysubsection{入れ子の原理}{sec:theory:ireko}

入れ子にする場合はこのようにする。


\mysection{実験方法}{sec:way}

ここに実験方法を書く。


%以下はプレレポートには必要ないのでコメントアウトしてある。適宜外すこと。
\ifpre\else
\mysection{使用器具}{sec:item}
%
今回の実験に使用した器具を、表\ref{tab:item}に示す。

\begin{aitems}{器具}{tab:item}
	\aline{パソコン}{メモリは100TBぐらいです!}{まいくろそふと}{嘘です}
	\aline{実験回路}{$R_a=100\mathrm{\Omega}$}{\――}{学校製}
\end{aitems}

\begin{table}[htbp]
	\caption{afafa}
	\label{afae}
	\begin{arraytabular}{rl}
		%一列目は右寄せなので、右側につめもの。
		%二列目は左寄せなので、左側に。
		%\xxrule 命令はbooktabsパッケージのものなので、エラー出たら全部\hline に置換してください。
		                                   \toprule
		R & P                           \\ \midrule
		1.5\phantom{0} & 3.14\phantom{00} \\
		2.25           & 3.1415           \\
		2.42\rlap{*}   & 3.1\phantom{000} \\ \bottomrule
		%出力の小数点位置がそろっているはず。
	\end{arraytabular}
\end{table}


\mysection{実験結果}{sec:result}

ここに実験結果を書く。


\mysection{考察}{sec:study}

ここに考察を書く。


%参考文献
% \cite{key}  ←を、参考文献を参照したい箇所に置く。
\begin{thebibliography}{New99}
\bibitem{アイテム} ここに書籍情報を書く
\bibitem[New]{新} 括弧内も変更可能。
%キャプションを変えるとずれるみたい。汚いから変えない方が無難。そのときは{New99}を{99}にする。
\end{thebibliography}

\fi
\end{document}
%%tipes
%\begin{figure}[htbp] %here,top,bottom,page. 図の希望位置
%	\centering %figure環境の中ではcenter環境よりこちらがいいらしい。
%	\includegraphics[width=10cm]{kairo.eps}
%	\caption{kairo test}
%	\label{mykairo}
%\end{figure}
%
%\renewcommand{\figurename}{Fig.} %図のキャプション装飾を変えたいとき。
%
%相互参照を適用するには、二度のコンパイルが必要。
